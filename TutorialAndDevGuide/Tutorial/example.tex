\section{Complete example}
\label{sec:complete_example}
This section describes a complete use case example in Copilot. We will be using
one of the provided examples as our code, and focus more on using Copilot
within a project.

The code implements a simple heater, which turns on when the temperature drops
below a certain point, and turns off if the temperature is too high. The
temperature is read from a sensor returns a byte, with a range of
$-50.0\degree$C to $100.0\degree$C. For ease of use, the monitor translates the
byte to a float within this range.

\subsection{Compiling Copilot} \label{sec:compiling}

Compiling a Copilot specification is straightforward. Currently Copilot
supports one back-end, \texttt{copilot-c99} that creates constant-space C99
code. Using the back-end is rather easy, as it just requires one to import it in
their Copilot specification file:

\begin{lstlisting}[language = Copilot]
import Copilot.Compile.C99
\end{lstlisting}

Importing the back-end provides us with the \texttt{compile}-function, which
takes a prefix as its first parameter and a \textit{reified} specification as
its second. When inside \texttt{ghci}, with our file loaded, we can generate
output code by executing:
\footnote{Two explanations are in order: (1) {\tt reify} allows sharing in the
expressions to be compiled~\cite{DSLExtract}, and {\tt >>=} is a higher-order
operator that takes the result of reification and ``feeds'' it to the compile
function.}
\begin{lstlisting}[language = Copilot]
reify spec >>= compile "heater"
\end{lstlisting}

This generates three output files:
\begin{itemize}
  \item \texttt{<prefix>.c}: C99 file containing the generated code and the
  \texttt{step()} function. This should be compiled by the C compiler, and
  included in the final binary.
  \item \texttt{<prefix>\_types.h}: Header file containing datatypes defined in the specification and used by Copilot internally. Copilot users can opt to include these definitions in their C code, to avoid having to manually define those datatypes and keep them in sync. However, when the Copilot specification is tied to an existing codebase, including this header from user code is not recommended as the definitions in it may conflict with C types defined elsewhere in the same codebase.
 \item \texttt{<prefix>.h}: Header providing the public interface to the
  monitor. This file should be included from your main project.
\end{itemize}

\subsection{Specification}
The code for this specification can be found in the \texttt{Examples} directory
of Copilot, or from the
repository\footnote{\url{https://github.com/Copilot-Language/Copilot/blob/master/Examples/Heater.hs}}.

\lstinputlisting[language = Copilot, frame = single, numbers = left]{Examples/Heater.hs}

The code should be self explanatory. Note that we opted to use a
\texttt{main}-function, which reifies and compiles the code for us.

On line 23 we can see the \texttt{ctemp}-stream, which is the temperature
translated to Celsius. Interestingly, we need to do a manual typecast from
\texttt{Word8} to \texttt{Float} using \texttt{unsafeCast}. This is a function
provided by Copilot that can cast a stream to a different type in an unsafe
manner, i.e. there may not be an exact representation of the value in both
types. For this code, it might happen that an integer value cannot be
represented exactly with a floating point.

\subsection{Generating C code}
Because we defined the \texttt{main}-function in our specification, generating
code is now really easy:
\begin{code}
$ runhaskell Heater.hs
\end{code}
This runs our Haskell code, with compiling a binary first. It runs the
\texttt{main}-function and generates the C code of our monitor. Note that it
created the files \texttt{heater.h}, \texttt{heater\_types.h} and \texttt{heater.c}, as defined by the
prefix passed to the \texttt{compile}-function.

The next step is as easy as compiling our C code together with the monitor. We
have opted to use GCC in a C99 mode with a extended set of warnings:
\begin{code}
$ gcc -Wall -std=c99 heater.c main.c -o heater
\end{code}

If the implementations for \texttt{heaton}, \texttt{heatoff} and
\texttt{readbyte} are provided somewhere, this should give us a nicely compiled
binary called \texttt{heater}. Running this will provide us with a system that
turns the heater on, when the temperature drops below $18.0\degree$C and turns
it off once it becomes higher than $21.0\degree$C.

\subsection{C Code}
After gernerating the C program for our monitor to connect to we can look at that file.

\lstinputlisting[language = c, frame = single, numbers = left]{Examples/heater.c}

For this code we left out the low-level details for interfacing with our
hardware. Let us look at a couple of interesting lines:

\begin{description}
  \item[Line 7] Here we include the header types files generated by out Copilot
  specification. This file must precede the inclusion of the generated header
  file on the next line. 
  \item[Line 8] Here we include the header file generated by our Copilot
  specification.
  \item[Line 10] Global variable that stores the raw output of the temperature
  sensor. This variable should be global, so it can be read from the code
  generate from our monitor.
  \item[Line 12-26] Functions that turn on and turn off the
  heater.
  \item[Line 28-39] Our infinite main-loop:
    \begin{description}
      \item[Line 31] Update our global temperature variable by reading it from
      the sensor.
      \item[Line 32-35/36-39] Check our guards that we defined above.
    \end{description}
\end{description}

As the code shows, the rate at which Copilot is updated is entirely up to the
programmer of the main C program. In this case it is updated as quick as
possible, but we could have opted to slow it down with a delay or a scheduler.
Theoretically there could be multiple calls to \texttt{step()} throughout the
program, but this complicates things and is highly discouraged.


