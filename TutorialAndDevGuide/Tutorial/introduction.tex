
\section{Introduction} \label{sec:introduction}


Neither formal verification nor testing can ensure system reliability.
%
Over 25 years ago, Butler and Finelli showed that testing alone cannot verify
the reliability of ultra-critical software~\cite{butler}.
%
\emph{Runtime verification} (RV)~\cite{monitors}, where monitors detect and
respond to property violations at runtime, has the potential to enable the safe
operation of safety-critical systems that are too complex to formally verify or
fully test.
%
Technically speaking, a RV monitor takes a logical specification $\phi$ and
execution trace $\tau$ of state information of the system under observation
(SUO) and decides whether $\tau$ satisfies $\phi$.
%
The \emph{Simplex Architecture}~\cite{simplex} provides a model architectural
pattern for RV, where a monitor checks that the executing SUO satisfies a
specification and, if the property is violated, the RV system will switch
control to a more conservative component that can be assured using conventional
means that \emph{steers} the system into a safe state.
%
\emph{High-assurance} RV provides an assured level of safety even when the SUO
itself cannot be verified by conventional means.
%%

Copilot\cite{PerezGoodloe20} is a RV framework for specifying and generating monitors for C programs
that are embedded into the functional programming language Haskell
\cite{PeytonJones02}.
%
 A working knowledge of Haskell is not necessary to use Copilot at a basic
level.
%
 However, knowledge of Haskell will significantly aid you in writing Copilot
specifications;  a variety of books and free web resources introduce Haskell.
%
  Copilot uses Haskell language extensions specific to the Glasgow Haskell
Compiler (GHC).

\subsection{Target Application Domain} \label{domain}


Copilot is a domain-specific language tailored to programming \emph{runtime
monitors} for \emph{hard real-time}, \emph{distributed}, \emph{reactive
systems}.
%
Briefly, a runtime monitor is a program that runs concurrently with a target
program of the SUO with the sole purpose of assuring that the target program
behaves in accordance with a pre-established specification.
%
 Copilot is a language for writing such specifications.
%

A reactive system is a system that responds continuously to its environment.
%
All data to and from a reactive system are communicated progressively during
execution.
%
Transformational systems, in contrast to reactive systems like Copilot,
transform data in a single pass and then terminate (examples are compilers and
numerical computation software).
%

A hard real-time system is a system that has a statically bounded execution
time and memory usage.
%
 Typically, hard real-time systems are used in mission-critical software, such
as avionics, medical equipment, and nuclear power plants; hence, occasional
dropouts in the response time or crashes are not tolerated.

A distributed system is a system that is layered out on multiple pieces of
hardware.
%
The distributed systems we consider are all synchronized, i.e., all components
agree on a shared global clock.


\subsection{Installation} \label{sec:install}

Before downloading the \href{https://github.com/Copilot-Language}{Copilot framework}, you must first install an
up-to-date version of GHC (the minimal required version is 8.6.4).
%

\noindent Copilot compiles to C code, therefor you must install a C compiler
such as GCC (\url{https://gcc.gnu.org/install/}). After having installed the
\href{https://www.haskell.org/}{Haskell Platform} and C compiler, Copilot can
be downloaded and installed in the following way:

\begin{itemize}
\item \textbf{Hackage: } Copilot is available from
\href{https://hackage.haskell.org/package/copilot-3.16#table-of-contents}{Hackage},
and the latest version can be installed easily:
\begin{code}
cabal v2-install --lib copilot
\end{code}

\end{itemize}


\subsection{Structure} \label{structure}

\noindent Copilot is distributed through a series of packages at Hackage:

\begin{itemize}
\item copilot-language: Contains the language front-end.
\item copilot-theorem: Contains extensions to the language for proving
properties about Copilot programs using various SMT solvers and model checkers.
\item copilot-core: Contains an intermediate representation for Copilot programs.
\item copilot-c99: A back-end for Copilot targeting C99.
\item copilot-libraries: A set of utility functions for Copilot, including a
clock-library, a linear temporal logic framework, a voting library, and a regular
expression framework.
\end{itemize}

All of the examples in this paper can be found at
\url{https://github.com/Copilot-Language/copilot-discussion/tree/master/TutorialAndDevGuide/Tutorial/Examples}

\subsection{Sampling} \label{sampling}
 The idea of sampling representative data from a large set of data  is well
established in data science and engineering.
%
 For instance, in digital signal processing, a signal such as music is sampled
at a high enough rate to obtain enough discrete points to represent the
physical sound wave.
%
 The fidelity of the recording is dependent on the sampling rate.
%
 Sampling a state variable of an executing program is similar, but variables
are rarely continuous signals so they lack the nice properties of continuity.
%
Monitoring based on sampling state-variables has historically been disregarded
as a runtime monitoring approach, for good reason: without the assumption of
synchrony between the monitor and observed software, monitoring via sampling
may lead to false positives and false negatives~\cite{DwyerDE08}.
%
 For example, consider the property $(0;1;1)^*$, written as a regular
expression, denoting the sequence of values a monitored variable may take.
%
 If the monitor samples the variable at an inappropriate time, then both false
negatives (the monitor erroneously rejects the sequence of values) and false
positives (the monitor erroneously accepts the sequence) are possible. 
%
 For example, if the actual sequence of values is $0,1,1,0,1,1$, then an
observation of $0,1,1,1,1$ is a false negative by skipping a value, and if the
actual sequence is $0,1,0,1,1$, then an observation of $0,1,1,0,1,1$ is a false
positive by sampling a value twice.
%


However, in a hard real-time context, sampling is a suitable strategy.
%
 Often, the purpose of real-time programs is to deliver output signals at a
predictable rate and properties of interest are generally data-flow oriented.
%
 In this context, and under the assumption that the monitor and the observed
program share a global clock and a static periodic schedule, while false
positives are possible, and false negatives are not.
%
 A false positive is possible, for example, if the program does not execute
according to its schedule but just happens to have the expected values when
sampled. 
%
 If a monitor samples an unacceptable sequence of values, then either the
program is in error, the monitor is in error, or they are not synchronized, all
of which are faults to be reported.
%

When deciding whether to use Copilot to monitor systems that are not hard
real-time, the user must determine if sampling is suitable to capture the
errors and faults of interest in the  SUO. In many cyber-physical systems, the
trace being monitored comes from sensors measuring physical attributes such as
GPS coordinates, air speed, and actuation signals.
%
 These continuous signals do not change abruptly so as long as it is sampled at
a suitable rate, that usually must be determined experimentally, sampling is
sufficient.
%

Most of the popular runtime monitoring frameworks utilize inline monitors in the
observed program to avoid the aforementioned problems with sampling.
%
 However, inlining monitors changes the real-time behavior of the observed
program, perhaps in unpredictable  ways.
%
Solutions that introduce such unpredictability are not a viable solution for
ultra-critical hard real-time systems.
%
 In a sampling-based approach, the monitor can be integrated as a separate
scheduled process during available time slices (this is made possible by
generating efficient constant-time monitors).
%
 Indeed, sampling-based monitors may even be scheduled on a separate processor
(albeit doing so requires additional synchronization mechanisms), ensuring time
and space partitioning from the observed programs.
%
 Such an architecture may even be necessary if the monitored program is
physically distributed.




