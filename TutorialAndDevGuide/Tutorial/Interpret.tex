\newpage 
\section{Interpreting}
\label{interpcompile}
The Copilot RV framework comes with both an interpreter and a
compiler. We will address compiling in \hyperref[sec:complete_example]{Section 4}
 with a complete example. 
\noindent To use the language, your Haskell module should contain the following import:
%
\begin{code}
import Language.Copilot
\end{code}
%
If you need to use functions defined in the Prelude that are redefined by
Copilot (e.g., arithmetic operators), import the Prelude qualified:
%
\begin{code}
import qualified Prelude as P
\end{code}

\subsection{Interpreting Copilot}
Instructive examples are provided in the  ./Examples directory 
for you to follow along with the examples in the tutorial. \texttt{Spec.hs} is the following Copilot program:
%
\lstinputlisting[language = Copilot, numbers = left]{Examples/Spec.hs}

\begin{description}
  \item[Line 1] Here we import Copilot Language so that we gain access to the
  front end language.
  \item[Line 3] Here we include the Prelude. Notice that we hide base Haskell
  syntax for functions that we define for use on streams. If this is not
  included you will get an \texttt{Ambiguous Occurrence} error. 
  \item[Line 5-12] Here we define data streams as input and output data streams. We
  go over defining functions as streams in Section 3 of this tutorial. 
  \item[Line 14-16] Here {\tt nats} is a stream of natural numbers, and {\tt
  evenNumber} and {\tt oddNumber} are the guard functions that take a stream and
  return whether the point-wise values are even or odd, respectively. The lists
  at the end of the trigger represent the values the trigger will output when the
  guard is true.

\end{description}

If we want to interpret the specification, we need to start the GHC Interpreter with the file as an argument:
%
\begin{lstlisting}
$ ghci Spec.hs
[1 of 1] Compiling Spec             ( Spec.hs, interpreted )
Ok, one module loaded.
ghci > 
\end{lstlisting}
%
This launches \texttt{ghci}, the Haskell interpreter, with \texttt{Spec.hs}
loaded. It provides us with a prompt, recognizable by the \texttt{>} sign. Let's
assume that our file contains one specification, called \texttt{spec}. We can
interpret this using the \texttt{interpret}-function:
\begin{lstlisting}[language = Copilot]
ghci > interpret 10 spec
\end{lstlisting}
%
The first argument to the function \emph{interpret} is the number of iterations
that we want to evaluate. The second argument is the specification (of type
{\tt Spec}) that we wish to interpret.

The interpreter outputs the values of the arguments passed to the trigger, if
its guard is true, and {\tt --} otherwise. We will discuss triggers in more
detail later, but for now, just know that they produce an output only when the
guard function is true. The output is as follows:
%
\begin{code}
trigger1:   trigger2:
(0,false)  --
--         (1)
(2,false)  --
--         (3)
(4,false)  --
--         (5)
(6,false)  --
--         (7)
(8,false)  --
--         (9)
\end{code}
%

Note that trigger1 outputs both the number and whether that number is odd,
while trigger2 only outputs the number. This output reflects the arguments
	passed to them. 


